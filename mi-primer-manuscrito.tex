\documentclass[spanish]{article}

\usepackage{arxiv}

\usepackage[utf8]{inputenc} % allow utf-8 input
\usepackage[T1]{fontenc}    % use 8-bit T1 fonts
\usepackage{lmodern}        % https://github.com/rstudio/rticles/issues/343
\usepackage{hyperref}       % hyperlinks
\usepackage{url}            % simple URL typesetting
\usepackage{booktabs}       % professional-quality tables
\usepackage{amsfonts}       % blackboard math symbols
\usepackage{nicefrac}       % compact symbols for 1/2, etc.
\usepackage{microtype}      % microtypography
\usepackage{graphicx}
\usepackage[backend=biber, style=apa, language=spanish]{biblatex}
\addbibresource{references.bib}
\usepackage{csquotes}
\usepackage[rgb,html,cmyk]{xcolor}


\title{TÍTULO EN ESPAÑOL \newline \textit{\large TÍTULO EN INGLÉS}}

\author{
    \parbox[t]{10cm}{\centering NOMBRE1-NOMBRE2 APELLIDO1-APELLIDO2 \\ \orcidlink{CÓDIGO ORCID}}
   \\
    Universidad Autónoma de Santo Domingo (UASD) \\
  Santo Domingo, República Dominicana \\
  \texttt{\href{mailto:ESTUDIANTE@SERVIDOR.COM}{\nolinkurl{ESTUDIANTE@SERVIDOR.COM}}} \\
  }


% tightlist command for lists without linebreak
\providecommand{\tightlist}{%
  \setlength{\itemsep}{0pt}\setlength{\parskip}{0pt}}



\usepackage{orcidlink} \usepackage{float} \usepackage[all]{nowidow} \usepackage[spanish]{babel} \usepackage{pdflscape} \renewcommand\spanishtablename{Tabla} \usepackage{xcolor} \usepackage{tabu} \renewcommand\tablename{Tabla} \renewcommand\figurename{Figura} \usepackage{xurl} \usepackage[left]{lineno} \linenumbers
\usepackage{booktabs}
\usepackage{longtable}
\usepackage{array}
\usepackage{multirow}
\usepackage{wrapfig}
\usepackage{float}
\usepackage{colortbl}
\usepackage{pdflscape}
\usepackage{tabu}
\usepackage{threeparttable}
\usepackage{threeparttablex}
\usepackage[normalem]{ulem}
\usepackage{makecell}
\usepackage{xcolor}
\begin{document}
\maketitle


\begin{resumen}
RESUMEN EN ESPAÑOL. NO INTRODUCIR RETORNOS
\end{resumen}

\palabrasclave{
    PALABRA CLAVE EN ESPAÑOL
   \and
    PALABRA CLAVE EN ESPAÑOL
   \and
    PALABRA CLAVE EN ESPAÑOL
   \and
    PALABRA CLAVE EN ESPAÑOL
  }

%\newpage

\begin{abstract}
RESUMEN EN INGLÉS. NO INTRODUCIR RETORNOS
\end{abstract}

\keywords{
    PALABRA CLAVE EN INGLÉS
   \and
    PALABRA CLAVE EN INGLÉS
   \and
    PALABRA CLAVE EN INGLÉS
   \and
    PALABRA CLAVE EN INGLÉS
  }

\begin{quote}
Nota 1 del Tali: el uso de MAYÚSCULAS en esta plantilla, no significa
``estoy gritando'', sino una convención para indicarte dónde debes
rellenar contenido. Tan pronto leas el mensaje escrito en mayúsculas,
bórralo para que no quede en tu entrega.
\end{quote}

\begin{quote}
Nota 2 del Tali: pide ayuda a inteligencia artificial, y recuerda
también usar el foro en caso de ``tranque''.
\end{quote}

\section*{``Hola Mundo''}\label{hola-mundo}
\addcontentsline{toc}{section}{``Hola Mundo''}

\begin{itemize}
\tightlist
\item
  INSERTA AQUÍ UN BLOQUE DE CÓDIGO PARA TU ``HOLA MUNDO''.
\end{itemize}

\section{Introducción}\label{introducciuxf3n}

\begin{itemize}
\item
  ESCRIBE AQUÍ LA SECCIÓN ``INTRODUCCIÓN''.
\item
  INCLUYE REFERENCIAS BIBLIOGRÁFICAS. LA SECCIÓN ``INTRODUCCIÓN'' ES UN
  BUEN LUGAR PARA ELLO. PARA CITAS DIRECTAS, USA
  \textcite{ETIQUETA_BIBTEX} (EN EL VÍDEO TUTORIAL APARECE DE OTRA
  MANERA, PERO USA ESTA MODALIDAD, POR SER MÁS SEGURA). INCLUYE LA CITA
  ENTRE CORCHETES, \autocite{ETIQUETA_BIBTEX}, SI QUIERES QUE, EN EL PDF
  TEJIDO, LA CITA APAREZCA ``(AUTOR, AÑO)''.
\item
  INSERTA AQUÍ UNA FIGURA DE ARCHIVO (E.G. UN ÚNICO ARCHIVO, SÓLO UNO,
  QUE MUESTRE, AUNQUE SEA PARCIALMENTE, LO QUE ENTREGASTE EN LA PRÁCTICA
  02. SI NO TIENES DICHO ARCHIVO, PUEDES USAR CUALQUIER OTRO ARCHIVO
  ALEGÓRICO AL TEMA O A LA PRÁCTICA).
\end{itemize}

\section{Materiales y métodos}\label{materiales-y-muxe9todos}

\begin{itemize}
\item
  ESCRIBE AQUÍ LA SECCIÓN ``MATERIALES Y MÉTODOS''.
\item
  INCLUYE REFERENCIAS BIBLIOGRÁFICAS. LA SECCIÓN ``MATERIALES Y
  MÉTODOS'' ES TAMBIÉN UN BUEN LUGAR PARA ELLO. PARA CITAS DIRECTAS, USA
  \textcite{ETIQUETA_BIBTEX} (EN EL VÍDEO TUTORIAL APARECE DE OTRA
  MANERA, PERO USA ESTA MODALIDAD, POR SER MÁS SEGURA). INCLUYE LA CITA
  ENTRE CORCHETES, \autocite{ETIQUETA_BIBTEX}, SI QUIERES QUE, EN EL PDF
  TEJIDO, LA CITA APAREZCA ``(AUTOR, AÑO)''.
\item
  INCLUYE AQUÍ LA TABLA DE DATOS DE FORMA ESTILIZADA, USANDO CÓDIGO DE R
  REPRODUCIBLE, COMANDO \texttt{knitr::kable}
\end{itemize}

\section{Resultados}\label{resultados}

\begin{itemize}
\item
  ESCRIBE AQUÍ LA SECCIÓN ``RESULTADOS''.
\item
  INSERTA AQUÍ UN DIAGRAMA DE CAJA GENERADO CON CÓDIGO DE R
  REPRODUCIBLE.
\item
  INSERTA AQUÍ UNA TABLA GENERADA MANUALMENTE EN MARKDOWN. POR EJEMPLO,
  GENERA UNA TABLA CON LOS RESULTADOS DE LA PRUEBA t DE STUDENT. EXISTEN
  HERRAMIENTAS EN LÍNEA PARA HACERLO; DE HECHO, EL PROPIO FORO DE LA
  ASIGNATURA TAMBIÉN ES CAPAZ DE CONVERTIR TABLAS DESDE EXCEL A
  MARKDOWN, BASTA CON COPIAR DESDE LA HOJA DE CÁLCULO Y PEGAR EN LA CAJA
  DE MENSAJES DEL FORO.
\end{itemize}

\section{Discusión}\label{discusiuxf3n}

\begin{itemize}
\item
  ESCRIBE AQUÍ LA SECCIÓN ``DISCUSIÓN''.
\item
  SI LO DESEAS, INCLUYE AQUÍ REFERENCIAS BIBLIOGRÁFICAS. LA SECCIÓN
  ``DISCUSIÓN'' ES TAMBIÉN UN BUEN LUGAR PARA ELLO. PARA CITAS DIRECTAS,
  USA \textcite{ETIQUETA_BIBTEX} (EN EL VÍDEO TUTORIAL APARECE DE OTRA
  MANERA, PERO USA ESTA MODALIDAD, POR SER MÁS SEGURA). INCLUYE LA CITA
  ENTRE CORCHETES, \autocite{ETIQUETA_BIBTEX}, SI QUIERES QUE, EN EL PDF
  TEJIDO, LA CITA APAREZCA ``(AUTOR, AÑO)''.
\end{itemize}

\printbibliography


\end{document}
